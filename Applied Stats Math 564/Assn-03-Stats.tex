% Options for packages loaded elsewhere
\PassOptionsToPackage{unicode}{hyperref}
\PassOptionsToPackage{hyphens}{url}
%
\documentclass[
]{article}
\usepackage{amsmath,amssymb}
\usepackage{lmodern}
\usepackage{iftex}
\ifPDFTeX
  \usepackage[T1]{fontenc}
  \usepackage[utf8]{inputenc}
  \usepackage{textcomp} % provide euro and other symbols
\else % if luatex or xetex
  \usepackage{unicode-math}
  \defaultfontfeatures{Scale=MatchLowercase}
  \defaultfontfeatures[\rmfamily]{Ligatures=TeX,Scale=1}
\fi
% Use upquote if available, for straight quotes in verbatim environments
\IfFileExists{upquote.sty}{\usepackage{upquote}}{}
\IfFileExists{microtype.sty}{% use microtype if available
  \usepackage[]{microtype}
  \UseMicrotypeSet[protrusion]{basicmath} % disable protrusion for tt fonts
}{}
\makeatletter
\@ifundefined{KOMAClassName}{% if non-KOMA class
  \IfFileExists{parskip.sty}{%
    \usepackage{parskip}
  }{% else
    \setlength{\parindent}{0pt}
    \setlength{\parskip}{6pt plus 2pt minus 1pt}}
}{% if KOMA class
  \KOMAoptions{parskip=half}}
\makeatother
\usepackage{xcolor}
\usepackage[margin=1in]{geometry}
\usepackage{color}
\usepackage{fancyvrb}
\newcommand{\VerbBar}{|}
\newcommand{\VERB}{\Verb[commandchars=\\\{\}]}
\DefineVerbatimEnvironment{Highlighting}{Verbatim}{commandchars=\\\{\}}
% Add ',fontsize=\small' for more characters per line
\usepackage{framed}
\definecolor{shadecolor}{RGB}{248,248,248}
\newenvironment{Shaded}{\begin{snugshade}}{\end{snugshade}}
\newcommand{\AlertTok}[1]{\textcolor[rgb]{0.94,0.16,0.16}{#1}}
\newcommand{\AnnotationTok}[1]{\textcolor[rgb]{0.56,0.35,0.01}{\textbf{\textit{#1}}}}
\newcommand{\AttributeTok}[1]{\textcolor[rgb]{0.77,0.63,0.00}{#1}}
\newcommand{\BaseNTok}[1]{\textcolor[rgb]{0.00,0.00,0.81}{#1}}
\newcommand{\BuiltInTok}[1]{#1}
\newcommand{\CharTok}[1]{\textcolor[rgb]{0.31,0.60,0.02}{#1}}
\newcommand{\CommentTok}[1]{\textcolor[rgb]{0.56,0.35,0.01}{\textit{#1}}}
\newcommand{\CommentVarTok}[1]{\textcolor[rgb]{0.56,0.35,0.01}{\textbf{\textit{#1}}}}
\newcommand{\ConstantTok}[1]{\textcolor[rgb]{0.00,0.00,0.00}{#1}}
\newcommand{\ControlFlowTok}[1]{\textcolor[rgb]{0.13,0.29,0.53}{\textbf{#1}}}
\newcommand{\DataTypeTok}[1]{\textcolor[rgb]{0.13,0.29,0.53}{#1}}
\newcommand{\DecValTok}[1]{\textcolor[rgb]{0.00,0.00,0.81}{#1}}
\newcommand{\DocumentationTok}[1]{\textcolor[rgb]{0.56,0.35,0.01}{\textbf{\textit{#1}}}}
\newcommand{\ErrorTok}[1]{\textcolor[rgb]{0.64,0.00,0.00}{\textbf{#1}}}
\newcommand{\ExtensionTok}[1]{#1}
\newcommand{\FloatTok}[1]{\textcolor[rgb]{0.00,0.00,0.81}{#1}}
\newcommand{\FunctionTok}[1]{\textcolor[rgb]{0.00,0.00,0.00}{#1}}
\newcommand{\ImportTok}[1]{#1}
\newcommand{\InformationTok}[1]{\textcolor[rgb]{0.56,0.35,0.01}{\textbf{\textit{#1}}}}
\newcommand{\KeywordTok}[1]{\textcolor[rgb]{0.13,0.29,0.53}{\textbf{#1}}}
\newcommand{\NormalTok}[1]{#1}
\newcommand{\OperatorTok}[1]{\textcolor[rgb]{0.81,0.36,0.00}{\textbf{#1}}}
\newcommand{\OtherTok}[1]{\textcolor[rgb]{0.56,0.35,0.01}{#1}}
\newcommand{\PreprocessorTok}[1]{\textcolor[rgb]{0.56,0.35,0.01}{\textit{#1}}}
\newcommand{\RegionMarkerTok}[1]{#1}
\newcommand{\SpecialCharTok}[1]{\textcolor[rgb]{0.00,0.00,0.00}{#1}}
\newcommand{\SpecialStringTok}[1]{\textcolor[rgb]{0.31,0.60,0.02}{#1}}
\newcommand{\StringTok}[1]{\textcolor[rgb]{0.31,0.60,0.02}{#1}}
\newcommand{\VariableTok}[1]{\textcolor[rgb]{0.00,0.00,0.00}{#1}}
\newcommand{\VerbatimStringTok}[1]{\textcolor[rgb]{0.31,0.60,0.02}{#1}}
\newcommand{\WarningTok}[1]{\textcolor[rgb]{0.56,0.35,0.01}{\textbf{\textit{#1}}}}
\usepackage{graphicx}
\makeatletter
\def\maxwidth{\ifdim\Gin@nat@width>\linewidth\linewidth\else\Gin@nat@width\fi}
\def\maxheight{\ifdim\Gin@nat@height>\textheight\textheight\else\Gin@nat@height\fi}
\makeatother
% Scale images if necessary, so that they will not overflow the page
% margins by default, and it is still possible to overwrite the defaults
% using explicit options in \includegraphics[width, height, ...]{}
\setkeys{Gin}{width=\maxwidth,height=\maxheight,keepaspectratio}
% Set default figure placement to htbp
\makeatletter
\def\fps@figure{htbp}
\makeatother
\setlength{\emergencystretch}{3em} % prevent overfull lines
\providecommand{\tightlist}{%
  \setlength{\itemsep}{0pt}\setlength{\parskip}{0pt}}
\setcounter{secnumdepth}{-\maxdimen} % remove section numbering
\ifLuaTeX
  \usepackage{selnolig}  % disable illegal ligatures
\fi
\IfFileExists{bookmark.sty}{\usepackage{bookmark}}{\usepackage{hyperref}}
\IfFileExists{xurl.sty}{\usepackage{xurl}}{} % add URL line breaks if available
\urlstyle{same} % disable monospaced font for URLs
\hypersetup{
  pdftitle={Asn 03 Stats},
  pdfauthor={Sumanth Donthula},
  hidelinks,
  pdfcreator={LaTeX via pandoc}}

\title{Asn 03 Stats}
\author{Sumanth Donthula}
\date{2022-10-09}

\begin{document}
\maketitle

Problem 1

1.a)

Yes, a linear relation is being observed in the data by scatter plot.

\begin{Shaded}
\begin{Highlighting}[]
\NormalTok{Data}\OtherTok{=}\FunctionTok{read.table}\NormalTok{(}\StringTok{"AS3Q1Data.txt"}\NormalTok{, }\AttributeTok{header =} \ConstantTok{FALSE}\NormalTok{, }\AttributeTok{sep =} \StringTok{""}\NormalTok{) }
\NormalTok{Y }\OtherTok{=}\NormalTok{ Data}\SpecialCharTok{$}\NormalTok{V1}
\NormalTok{X }\OtherTok{=}\NormalTok{ Data}\SpecialCharTok{$}\NormalTok{V2}

\FunctionTok{plot}\NormalTok{(X,Y,}\AttributeTok{xlab=}\StringTok{"Sales"}\NormalTok{,}\AttributeTok{ylab=}\StringTok{"Sales Year"}\NormalTok{)}
\end{Highlighting}
\end{Shaded}

\includegraphics{Assn-03-Stats_files/figure-latex/unnamed-chunk-1-1.pdf}

\begin{Shaded}
\begin{Highlighting}[]
\NormalTok{lm1}\OtherTok{=}\FunctionTok{lm}\NormalTok{(Y}\SpecialCharTok{\textasciitilde{}}\NormalTok{X)}
\NormalTok{lm1}
\end{Highlighting}
\end{Shaded}

\begin{verbatim}
## 
## Call:
## lm(formula = Y ~ X)
## 
## Coefficients:
## (Intercept)            X  
##       91.56        32.50
\end{verbatim}

1.b)

By looking at the box cox plot a lambda of 0.5 is suggested. By SSE box
cox plot it is evident that SSE is miniumum at lambda=0.5.

\begin{Shaded}
\begin{Highlighting}[]
\FunctionTok{resid}\NormalTok{(lm1)}
\end{Highlighting}
\end{Shaded}

\begin{verbatim}
##           1           2           3           4           5           6 
##   6.4363636  10.9393939   5.4424242 -11.0545455  -0.5515152 -22.0484848 
##           7           8           9          10 
##  -3.5454545 -19.0424242  22.4606061  10.9636364
\end{verbatim}

\begin{Shaded}
\begin{Highlighting}[]
\FunctionTok{library}\NormalTok{(MASS)}
\FunctionTok{boxcox}\NormalTok{(Y}\SpecialCharTok{\textasciitilde{}}\NormalTok{X)}
\end{Highlighting}
\end{Shaded}

\includegraphics{Assn-03-Stats_files/figure-latex/unnamed-chunk-2-1.pdf}

\begin{Shaded}
\begin{Highlighting}[]
\FunctionTok{boxcox}\NormalTok{(Y}\SpecialCharTok{\textasciitilde{}}\NormalTok{X,}\FunctionTok{seq}\NormalTok{(}\DecValTok{0}\NormalTok{,}\DecValTok{1}\NormalTok{,}\FloatTok{0.01}\NormalTok{))}
\end{Highlighting}
\end{Shaded}

\includegraphics{Assn-03-Stats_files/figure-latex/unnamed-chunk-2-2.pdf}

\begin{Shaded}
\begin{Highlighting}[]
\FunctionTok{library}\NormalTok{(}\StringTok{\textquotesingle{}ALSM\textquotesingle{}}\NormalTok{)}
\end{Highlighting}
\end{Shaded}

\begin{verbatim}
## Loading required package: leaps
\end{verbatim}

\begin{verbatim}
## Loading required package: SuppDists
\end{verbatim}

\begin{verbatim}
## Loading required package: car
\end{verbatim}

\begin{verbatim}
## Loading required package: carData
\end{verbatim}

\begin{Shaded}
\begin{Highlighting}[]
\FunctionTok{boxcox.sse}\NormalTok{(X,Y)}
\end{Highlighting}
\end{Shaded}

\includegraphics{Assn-03-Stats_files/figure-latex/unnamed-chunk-2-3.pdf}

\begin{verbatim}
##    lambda        SSE
## 1    -2.0 57788.3511
## 2    -1.9 50489.6939
## 3    -1.8 44054.1816
## 4    -1.7 38379.8733
## 5    -1.6 33377.2323
## 6    -1.5 28967.6103
## 7    -1.4 25081.9206
## 8    -1.3 21659.4777
## 9    -1.2 18646.9811
## 10   -1.1 15997.6263
## 11   -1.0 13670.3269
## 12   -0.9 11629.0334
## 13   -0.8  9842.1378
## 14   -0.7  8281.9520
## 15   -0.6  6924.2528
## 16   -0.5  5747.8831
## 17   -0.4  4734.4047
## 18   -0.3  3867.7951
## 19   -0.2  3134.1829
## 20   -0.1  2521.6190
## 41    0.0  2019.8767
## 21    0.1  1620.2804
## 22    0.2  1315.5569
## 23    0.3  1099.7093
## 24    0.4   967.9088
## 25    0.5   916.4048
## 26    0.6   942.4498
## 27    0.7  1044.2384
## 28    0.8  1220.8598
## 29    0.9  1472.2614
## 30    1.0  1799.2242
## 31    1.1  2203.3483
## 32    1.2  2687.0483
## 33    1.3  3253.5588
## 34    1.4  3906.9485
## 35    1.5  4652.1447
## 36    1.6  5494.9660
## 37    1.7  6442.1649
## 38    1.8  7501.4808
## 39    1.9  8681.7016
## 40    2.0  9992.7371
\end{verbatim}

1.c)

The linear relation function is : 10.26093+1.076X

\begin{Shaded}
\begin{Highlighting}[]
\NormalTok{Yroot}\OtherTok{=}\NormalTok{Y}\SpecialCharTok{\^{}}\FloatTok{0.5}
\NormalTok{lm2}\OtherTok{=}\FunctionTok{lm}\NormalTok{(Yroot}\SpecialCharTok{\textasciitilde{}}\NormalTok{X)}
\NormalTok{lm2}
\end{Highlighting}
\end{Shaded}

\begin{verbatim}
## 
## Call:
## lm(formula = Yroot ~ X)
## 
## Coefficients:
## (Intercept)            X  
##      10.261        1.076
\end{verbatim}

1.d)

Yes, the linear regression seems a good fit on the transformed data.

\begin{Shaded}
\begin{Highlighting}[]
\FunctionTok{plot}\NormalTok{(X,Yroot)}\SpecialCharTok{+}\FunctionTok{abline}\NormalTok{(}\FloatTok{10.261}\NormalTok{,}\FloatTok{1.076}\NormalTok{,}\AttributeTok{col=}\DecValTok{3}\NormalTok{)}
\end{Highlighting}
\end{Shaded}

\includegraphics{Assn-03-Stats_files/figure-latex/unnamed-chunk-4-1.pdf}

\begin{verbatim}
## integer(0)
\end{verbatim}

1.e) Sum of residuals by looking at the residual plot is almost 0 which
supports this transformation

From Qq plot, qq line does not line up perfectly but it appears to be in
linear relation. So we conclude that residuals are normally distributed.

\begin{Shaded}
\begin{Highlighting}[]
\NormalTok{residual}\OtherTok{=}\NormalTok{lm2}\SpecialCharTok{$}\NormalTok{residuals}
\NormalTok{residual}
\end{Highlighting}
\end{Shaded}

\begin{verbatim}
##           1           2           3           4           5           6 
## -0.36143656  0.28172678  0.31440703 -0.14814273  0.29997018 -0.41084412 
##           7           8           9          10 
##  0.10392174 -0.47446579  0.46781397 -0.07295049
\end{verbatim}

\begin{Shaded}
\begin{Highlighting}[]
\NormalTok{predictors}\OtherTok{=}\NormalTok{lm2}\SpecialCharTok{$}\NormalTok{fitted.values}
\FunctionTok{plot}\NormalTok{(predictors,residual)}
\end{Highlighting}
\end{Shaded}

\includegraphics{Assn-03-Stats_files/figure-latex/unnamed-chunk-5-1.pdf}

\begin{Shaded}
\begin{Highlighting}[]
\FunctionTok{plot}\NormalTok{(lm2,}\AttributeTok{which=}\FunctionTok{c}\NormalTok{(}\DecValTok{1}\NormalTok{,}\DecValTok{2}\NormalTok{))}
\end{Highlighting}
\end{Shaded}

\includegraphics{Assn-03-Stats_files/figure-latex/unnamed-chunk-5-2.pdf}
\includegraphics{Assn-03-Stats_files/figure-latex/unnamed-chunk-5-3.pdf}

1.f)

The estimated function in original units is:

Ybar=(10.261+1.076X)2

Problem 2:

2.a)

The confidence intervals are

CI45 : 98.6309, 106.9691 CI55 : 88.11124, 93.68876 CI65 : 76.20837,
81.79163

\begin{Shaded}
\begin{Highlighting}[]
\NormalTok{Data2}\OtherTok{=}\FunctionTok{read.table}\NormalTok{(}\StringTok{"AS32Data.txt"}\NormalTok{, }\AttributeTok{header =} \ConstantTok{FALSE}\NormalTok{, }\AttributeTok{sep =} \StringTok{""}\NormalTok{) }
\NormalTok{Y }\OtherTok{=}\NormalTok{ Data2}\SpecialCharTok{$}\NormalTok{V1}
\NormalTok{X }\OtherTok{=}\NormalTok{ Data2}\SpecialCharTok{$}\NormalTok{V2}
\NormalTok{n}\OtherTok{=}\FunctionTok{length}\NormalTok{(X)}
\NormalTok{Xf }\OtherTok{=}\FunctionTok{cbind}\NormalTok{(}\FunctionTok{rep}\NormalTok{(}\DecValTok{1}\NormalTok{,n),X)}


\NormalTok{Ymat}\OtherTok{=}\FunctionTok{as.matrix}\NormalTok{(Y)}
\NormalTok{Xmat}\OtherTok{=}\FunctionTok{as.matrix}\NormalTok{(Xf)}


\NormalTok{lm}\OtherTok{=}\FunctionTok{lm}\NormalTok{(Y}\SpecialCharTok{\textasciitilde{}}\NormalTok{X)}
\NormalTok{lm}
\end{Highlighting}
\end{Shaded}

\begin{verbatim}
## 
## Call:
## lm(formula = Y ~ X)
## 
## Coefficients:
## (Intercept)            X  
##      156.35        -1.19
\end{verbatim}

\begin{Shaded}
\begin{Highlighting}[]
\FunctionTok{anova}\NormalTok{(lm)}
\end{Highlighting}
\end{Shaded}

\begin{verbatim}
## Analysis of Variance Table
## 
## Response: Y
##           Df  Sum Sq Mean Sq F value    Pr(>F)    
## X          1 11627.5 11627.5  174.06 < 2.2e-16 ***
## Residuals 58  3874.4    66.8                      
## ---
## Signif. codes:  0 '***' 0.001 '**' 0.01 '*' 0.05 '.' 0.1 ' ' 1
\end{verbatim}

\begin{Shaded}
\begin{Highlighting}[]
\FunctionTok{summary}\NormalTok{(lm)}
\end{Highlighting}
\end{Shaded}

\begin{verbatim}
## 
## Call:
## lm(formula = Y ~ X)
## 
## Residuals:
##      Min       1Q   Median       3Q      Max 
## -16.1368  -6.1968  -0.5969   6.7607  23.4731 
## 
## Coefficients:
##             Estimate Std. Error t value Pr(>|t|)    
## (Intercept) 156.3466     5.5123   28.36   <2e-16 ***
## X            -1.1900     0.0902  -13.19   <2e-16 ***
## ---
## Signif. codes:  0 '***' 0.001 '**' 0.01 '*' 0.05 '.' 0.1 ' ' 1
## 
## Residual standard error: 8.173 on 58 degrees of freedom
## Multiple R-squared:  0.7501, Adjusted R-squared:  0.7458 
## F-statistic: 174.1 on 1 and 58 DF,  p-value: < 2.2e-16
\end{verbatim}

\begin{Shaded}
\begin{Highlighting}[]
\NormalTok{sse }\OtherTok{=} \FunctionTok{sum}\NormalTok{((}\FunctionTok{fitted}\NormalTok{(lm) }\SpecialCharTok{{-}}\NormalTok{ Y)}\SpecialCharTok{\^{}}\DecValTok{2}\NormalTok{)}

\NormalTok{Sigmasquare}\OtherTok{=}\NormalTok{sse}\SpecialCharTok{/}\NormalTok{n}\DecValTok{{-}2}

\CommentTok{\#Working hotelling method}
\NormalTok{X1h}\OtherTok{=}\FunctionTok{c}\NormalTok{(}\DecValTok{1}\NormalTok{,}\DecValTok{45}\NormalTok{)}
\NormalTok{X2h}\OtherTok{=}\FunctionTok{c}\NormalTok{(}\DecValTok{1}\NormalTok{,}\DecValTok{55}\NormalTok{)}
\NormalTok{X3h}\OtherTok{=}\FunctionTok{c}\NormalTok{(}\DecValTok{1}\NormalTok{,}\DecValTok{65}\NormalTok{)}

\NormalTok{Yhat1}\OtherTok{=}\FloatTok{156.35{-}1.19}\SpecialCharTok{*}\DecValTok{45}
\NormalTok{Yhat2}\OtherTok{=}\FloatTok{156.35{-}1.19}\SpecialCharTok{*}\DecValTok{55}
\NormalTok{Yhat3}\OtherTok{=}\FloatTok{156.35{-}1.19}\SpecialCharTok{*}\DecValTok{65}

\NormalTok{measure1}\OtherTok{=}\NormalTok{(Sigmasquare}\SpecialCharTok{*}\FunctionTok{t}\NormalTok{(X1h)}\SpecialCharTok{\%*\%}\FunctionTok{solve}\NormalTok{(}\FunctionTok{t}\NormalTok{(Xmat)}\SpecialCharTok{\%*\%}\NormalTok{Xmat)}\SpecialCharTok{\%*\%}\NormalTok{X1h)}\SpecialCharTok{\^{}}\FloatTok{0.5}

\NormalTok{measure2}\OtherTok{=}\NormalTok{(Sigmasquare}\SpecialCharTok{*}\FunctionTok{t}\NormalTok{(X2h)}\SpecialCharTok{\%*\%}\FunctionTok{solve}\NormalTok{(}\FunctionTok{t}\NormalTok{(Xmat)}\SpecialCharTok{\%*\%}\NormalTok{Xmat)}\SpecialCharTok{\%*\%}\NormalTok{X2h)}\SpecialCharTok{\^{}}\FloatTok{0.5}

\NormalTok{measure3}\OtherTok{=}\NormalTok{(Sigmasquare}\SpecialCharTok{*}\FunctionTok{t}\NormalTok{(X3h)}\SpecialCharTok{\%*\%}\FunctionTok{solve}\NormalTok{(}\FunctionTok{t}\NormalTok{(Xmat)}\SpecialCharTok{\%*\%}\NormalTok{Xmat)}\SpecialCharTok{\%*\%}\NormalTok{X3h)}\SpecialCharTok{\^{}}\FloatTok{0.5}

\NormalTok{W }\OtherTok{=} \FunctionTok{sqrt}\NormalTok{(}\DecValTok{2} \SpecialCharTok{*} \FunctionTok{qf}\NormalTok{(}\AttributeTok{p =} \FloatTok{0.95}\NormalTok{, }\AttributeTok{df1 =} \DecValTok{2}\NormalTok{, }\AttributeTok{df2 =}\NormalTok{ n }\SpecialCharTok{{-}} \DecValTok{2}\NormalTok{))}
\NormalTok{W}
\end{Highlighting}
\end{Shaded}

\begin{verbatim}
## [1] 2.512342
\end{verbatim}

\begin{Shaded}
\begin{Highlighting}[]
\NormalTok{conf1up}\OtherTok{=}\NormalTok{Yhat1}\SpecialCharTok{+}\NormalTok{W}\SpecialCharTok{*}\NormalTok{measure1 }
\NormalTok{conf1lo}\OtherTok{=}\NormalTok{Yhat1}\SpecialCharTok{{-}}\NormalTok{W}\SpecialCharTok{*}\NormalTok{measure1  }
\NormalTok{conf2up}\OtherTok{=}\NormalTok{Yhat2}\SpecialCharTok{+}\NormalTok{W}\SpecialCharTok{*}\NormalTok{measure2 }
\NormalTok{conf2lo}\OtherTok{=}\NormalTok{Yhat2}\SpecialCharTok{{-}}\NormalTok{W}\SpecialCharTok{*}\NormalTok{measure2  }
\NormalTok{conf3up}\OtherTok{=}\NormalTok{Yhat3}\SpecialCharTok{+}\NormalTok{W}\SpecialCharTok{*}\NormalTok{measure3  }
\NormalTok{conf3lo}\OtherTok{=}\NormalTok{Yhat3}\SpecialCharTok{{-}}\NormalTok{W}\SpecialCharTok{*}\NormalTok{measure3  }

\NormalTok{conf45}\OtherTok{=}\FunctionTok{cbind}\NormalTok{(conf1lo,conf1up)}
\NormalTok{conf45}
\end{Highlighting}
\end{Shaded}

\begin{verbatim}
##         [,1]     [,2]
## [1,] 98.6309 106.9691
\end{verbatim}

\begin{Shaded}
\begin{Highlighting}[]
\NormalTok{conf55}\OtherTok{=}\FunctionTok{cbind}\NormalTok{(conf2lo,conf2up)}
\NormalTok{conf55}
\end{Highlighting}
\end{Shaded}

\begin{verbatim}
##          [,1]     [,2]
## [1,] 88.11124 93.68876
\end{verbatim}

\begin{Shaded}
\begin{Highlighting}[]
\NormalTok{conf65}\OtherTok{=}\FunctionTok{cbind}\NormalTok{(conf3lo,conf3up)}
\NormalTok{conf65}
\end{Highlighting}
\end{Shaded}

\begin{verbatim}
##          [,1]     [,2]
## [1,] 76.20837 81.79163
\end{verbatim}

2.b) NO the working hotel model is not the most efficient one as its
range is wider compared to normal t distributions confidence interval.

For example here t is 1.67 where as w is 2.51 the band will be larger.

\begin{Shaded}
\begin{Highlighting}[]
\NormalTok{t }\OtherTok{=} \FunctionTok{qt}\NormalTok{(}\FloatTok{0.95}\NormalTok{,}\FunctionTok{nrow}\NormalTok{(Data2) }\SpecialCharTok{{-}} \DecValTok{2}\NormalTok{)}
\NormalTok{t}
\end{Highlighting}
\end{Shaded}

\begin{verbatim}
## [1] 1.671553
\end{verbatim}

2.c)

The confidence intervals are

CI48 : 95.62575, 102.8342 CI59 : 83.61339, 88.66661 CI74 : 64.3607,
72.2193

\begin{Shaded}
\begin{Highlighting}[]
\NormalTok{BX1h}\OtherTok{=}\FunctionTok{c}\NormalTok{(}\DecValTok{1}\NormalTok{,}\DecValTok{48}\NormalTok{)}
\NormalTok{BX2h}\OtherTok{=}\FunctionTok{c}\NormalTok{(}\DecValTok{1}\NormalTok{,}\DecValTok{59}\NormalTok{)}
\NormalTok{BX3h}\OtherTok{=}\FunctionTok{c}\NormalTok{(}\DecValTok{1}\NormalTok{,}\DecValTok{74}\NormalTok{)}

\NormalTok{Yhat11}\OtherTok{=}\FloatTok{156.35{-}1.19}\SpecialCharTok{*}\DecValTok{48}
\NormalTok{Yhat21}\OtherTok{=}\FloatTok{156.35{-}1.19}\SpecialCharTok{*}\DecValTok{59}
\NormalTok{Yhat31}\OtherTok{=}\FloatTok{156.35{-}1.19}\SpecialCharTok{*}\DecValTok{74}


\NormalTok{Bmeasure1}\OtherTok{=}\NormalTok{(Sigmasquare}\SpecialCharTok{*}\FunctionTok{t}\NormalTok{(BX1h)}\SpecialCharTok{\%*\%}\FunctionTok{solve}\NormalTok{(}\FunctionTok{t}\NormalTok{(Xmat)}\SpecialCharTok{\%*\%}\NormalTok{Xmat)}\SpecialCharTok{\%*\%}\NormalTok{BX1h)}\SpecialCharTok{\^{}}\FloatTok{0.5}

\NormalTok{Bmeasure2}\OtherTok{=}\NormalTok{(Sigmasquare}\SpecialCharTok{*}\FunctionTok{t}\NormalTok{(BX2h)}\SpecialCharTok{\%*\%}\FunctionTok{solve}\NormalTok{(}\FunctionTok{t}\NormalTok{(Xmat)}\SpecialCharTok{\%*\%}\NormalTok{Xmat)}\SpecialCharTok{\%*\%}\NormalTok{BX2h)}\SpecialCharTok{\^{}}\FloatTok{0.5}

\NormalTok{Bmeasure3}\OtherTok{=}\NormalTok{(Sigmasquare}\SpecialCharTok{*}\FunctionTok{t}\NormalTok{(BX3h)}\SpecialCharTok{\%*\%}\FunctionTok{solve}\NormalTok{(}\FunctionTok{t}\NormalTok{(Xmat)}\SpecialCharTok{\%*\%}\NormalTok{Xmat)}\SpecialCharTok{\%*\%}\NormalTok{BX3h)}\SpecialCharTok{\^{}}\FloatTok{0.5}

\NormalTok{B }\OtherTok{=} \FunctionTok{qt}\NormalTok{(}\DecValTok{1}\FloatTok{{-}0.05}\SpecialCharTok{/}\NormalTok{(}\DecValTok{2} \SpecialCharTok{*} \DecValTok{3}\NormalTok{), n }\SpecialCharTok{{-}} \DecValTok{2}\NormalTok{)}

\NormalTok{conf11up}\OtherTok{=}\NormalTok{Yhat11}\SpecialCharTok{+}\NormalTok{B}\SpecialCharTok{*}\NormalTok{Bmeasure1}
\NormalTok{conf11lo}\OtherTok{=}\NormalTok{Yhat11}\SpecialCharTok{{-}}\NormalTok{B}\SpecialCharTok{*}\NormalTok{Bmeasure1  }
\NormalTok{conf22up}\OtherTok{=}\NormalTok{Yhat21}\SpecialCharTok{+}\NormalTok{B}\SpecialCharTok{*}\NormalTok{Bmeasure2 }
\NormalTok{conf22lo}\OtherTok{=}\NormalTok{Yhat21}\SpecialCharTok{{-}}\NormalTok{B}\SpecialCharTok{*}\NormalTok{Bmeasure2  }
\NormalTok{conf33up}\OtherTok{=}\NormalTok{Yhat31}\SpecialCharTok{+}\NormalTok{B}\SpecialCharTok{*}\NormalTok{Bmeasure3  }
\NormalTok{conf33lo}\OtherTok{=}\NormalTok{Yhat31}\SpecialCharTok{{-}}\NormalTok{B}\SpecialCharTok{*}\NormalTok{Bmeasure3}
\NormalTok{conf48}\OtherTok{=}\FunctionTok{cbind}\NormalTok{(conf11lo,conf11up)}
\NormalTok{conf48}
\end{Highlighting}
\end{Shaded}

\begin{verbatim}
##          [,1]     [,2]
## [1,] 95.62575 102.8342
\end{verbatim}

\begin{Shaded}
\begin{Highlighting}[]
\NormalTok{conf59}\OtherTok{=}\FunctionTok{cbind}\NormalTok{(conf22lo,conf22up)}
\NormalTok{conf59}
\end{Highlighting}
\end{Shaded}

\begin{verbatim}
##          [,1]     [,2]
## [1,] 83.61339 88.66661
\end{verbatim}

\begin{Shaded}
\begin{Highlighting}[]
\NormalTok{conf74}\OtherTok{=}\FunctionTok{cbind}\NormalTok{(conf33lo,conf33up)}
\NormalTok{conf74}
\end{Highlighting}
\end{Shaded}

\begin{verbatim}
##         [,1]    [,2]
## [1,] 64.3607 72.2193
\end{verbatim}

Problem 3

3.a)

The observations from plots provided that there are no outliers and the
distribution of each variable is normal.

Correlation matrix shows Y and X1 have significant positive correlation,
Y and X2 are positively correlated, but less than Y and X1 and there's
no correlation between X1 and X2.

\begin{Shaded}
\begin{Highlighting}[]
\NormalTok{Data3}\OtherTok{=}\FunctionTok{read.table}\NormalTok{(}\StringTok{"AS33Data.txt"}\NormalTok{, }\AttributeTok{header =} \ConstantTok{FALSE}\NormalTok{, }\AttributeTok{sep =} \StringTok{""}\NormalTok{) }
\NormalTok{Y }\OtherTok{=}\NormalTok{ Data3}\SpecialCharTok{$}\NormalTok{V1}
\NormalTok{X1}\OtherTok{=}\NormalTok{ Data3}\SpecialCharTok{$}\NormalTok{V2}
\NormalTok{X2}\OtherTok{=}\NormalTok{Data3}\SpecialCharTok{$}\NormalTok{V3}


\FunctionTok{pairs}\NormalTok{(}\SpecialCharTok{\textasciitilde{}}\NormalTok{Y}\SpecialCharTok{+}\NormalTok{X1}\SpecialCharTok{+}\NormalTok{X2)}
\end{Highlighting}
\end{Shaded}

\includegraphics{Assn-03-Stats_files/figure-latex/unnamed-chunk-9-1.pdf}

\begin{Shaded}
\begin{Highlighting}[]
\FunctionTok{colnames}\NormalTok{(Data3)}\OtherTok{=}\FunctionTok{c}\NormalTok{(}\StringTok{"Y"}\NormalTok{,}\StringTok{"X1"}\NormalTok{,}\StringTok{"X2"}\NormalTok{)}
\FunctionTok{cor}\NormalTok{(Data3)}
\end{Highlighting}
\end{Shaded}

\begin{verbatim}
##            Y        X1        X2
## Y  1.0000000 0.8923929 0.3945807
## X1 0.8923929 1.0000000 0.0000000
## X2 0.3945807 0.0000000 1.0000000
\end{verbatim}

3.b)The regression model is Y= 37.65 + 4.425X1 + 4.375X2. Holding the
other variable constant, Increasing one unit of X1 leads to an increase
in the brand liking by 4.425, and holding X1 constant, an one unit
increase in X2 leads to an increase of the brand by 4.375.

\begin{Shaded}
\begin{Highlighting}[]
\NormalTok{Lmodel}\OtherTok{=}\FunctionTok{lm}\NormalTok{(Y}\SpecialCharTok{\textasciitilde{}}\NormalTok{X1}\SpecialCharTok{+}\NormalTok{X2)}
\FunctionTok{summary}\NormalTok{(Lmodel)}
\end{Highlighting}
\end{Shaded}

\begin{verbatim}
## 
## Call:
## lm(formula = Y ~ X1 + X2)
## 
## Residuals:
##    Min     1Q Median     3Q    Max 
## -4.400 -1.762  0.025  1.587  4.200 
## 
## Coefficients:
##             Estimate Std. Error t value Pr(>|t|)    
## (Intercept)  37.6500     2.9961  12.566 1.20e-08 ***
## X1            4.4250     0.3011  14.695 1.78e-09 ***
## X2            4.3750     0.6733   6.498 2.01e-05 ***
## ---
## Signif. codes:  0 '***' 0.001 '**' 0.01 '*' 0.05 '.' 0.1 ' ' 1
## 
## Residual standard error: 2.693 on 13 degrees of freedom
## Multiple R-squared:  0.9521, Adjusted R-squared:  0.9447 
## F-statistic: 129.1 on 2 and 13 DF,  p-value: 2.658e-09
\end{verbatim}

\begin{Shaded}
\begin{Highlighting}[]
\CommentTok{\#Y=37.65+4.25X1+4.375X2}
\end{Highlighting}
\end{Shaded}

3.c)

There are no outliers in the residuals and errors are normally
distributed.

\begin{Shaded}
\begin{Highlighting}[]
\NormalTok{Lmodel}\SpecialCharTok{$}\NormalTok{residual}
\end{Highlighting}
\end{Shaded}

\begin{verbatim}
##     1     2     3     4     5     6     7     8     9    10    11    12    13 
## -0.10  0.15 -3.10  3.15 -0.95 -1.70 -1.95  1.30  1.20 -1.55  4.20  2.45 -2.65 
##    14    15    16 
## -4.40  3.35  0.60
\end{verbatim}

\begin{Shaded}
\begin{Highlighting}[]
\FunctionTok{boxplot}\NormalTok{(Lmodel}\SpecialCharTok{$}\NormalTok{residual)}
\end{Highlighting}
\end{Shaded}

\includegraphics{Assn-03-Stats_files/figure-latex/unnamed-chunk-11-1.pdf}

3.d)

Based on the plots, we observe that residuals are random and almost
normally distributed with mean 0.

\begin{Shaded}
\begin{Highlighting}[]
\NormalTok{Yhat}\OtherTok{=}\FloatTok{37.65+4.25}\SpecialCharTok{*}\NormalTok{X1}\FloatTok{+4.375}\SpecialCharTok{*}\NormalTok{X2}

\FunctionTok{plot}\NormalTok{(Yhat,Lmodel}\SpecialCharTok{$}\NormalTok{residual)}
\end{Highlighting}
\end{Shaded}

\includegraphics{Assn-03-Stats_files/figure-latex/unnamed-chunk-12-1.pdf}

\begin{Shaded}
\begin{Highlighting}[]
\FunctionTok{plot}\NormalTok{(X1,Lmodel}\SpecialCharTok{$}\NormalTok{residual)}
\end{Highlighting}
\end{Shaded}

\includegraphics{Assn-03-Stats_files/figure-latex/unnamed-chunk-12-2.pdf}

\begin{Shaded}
\begin{Highlighting}[]
\FunctionTok{plot}\NormalTok{(X2,Lmodel}\SpecialCharTok{$}\NormalTok{residual)}
\end{Highlighting}
\end{Shaded}

\includegraphics{Assn-03-Stats_files/figure-latex/unnamed-chunk-12-3.pdf}

\begin{Shaded}
\begin{Highlighting}[]
\FunctionTok{plot}\NormalTok{(X1}\SpecialCharTok{*}\NormalTok{X2,Lmodel}\SpecialCharTok{$}\NormalTok{residual)}
\end{Highlighting}
\end{Shaded}

\includegraphics{Assn-03-Stats_files/figure-latex/unnamed-chunk-12-4.pdf}

\begin{Shaded}
\begin{Highlighting}[]
\FunctionTok{qqnorm}\NormalTok{(Lmodel}\SpecialCharTok{$}\NormalTok{residual)}
\FunctionTok{qqline}\NormalTok{(Lmodel}\SpecialCharTok{$}\NormalTok{residual)}
\end{Highlighting}
\end{Shaded}

\includegraphics{Assn-03-Stats_files/figure-latex/unnamed-chunk-12-5.pdf}
3.e)

Ho: Error variance is constant Ha: Error variance is not constant

The p value of the Breusch Pagan test is 0.3599 which is greater than
alpha 0.05 so we reject the null hypothesis.

\begin{Shaded}
\begin{Highlighting}[]
\FunctionTok{library}\NormalTok{(lmtest)}
\end{Highlighting}
\end{Shaded}

\begin{verbatim}
## Loading required package: zoo
\end{verbatim}

\begin{verbatim}
## 
## Attaching package: 'zoo'
\end{verbatim}

\begin{verbatim}
## The following objects are masked from 'package:base':
## 
##     as.Date, as.Date.numeric
\end{verbatim}

\begin{Shaded}
\begin{Highlighting}[]
\NormalTok{Lmodel2}\OtherTok{=}\FunctionTok{lm}\NormalTok{(}\FunctionTok{log}\NormalTok{((Lmodel}\SpecialCharTok{$}\NormalTok{residuals)}\SpecialCharTok{\^{}}\DecValTok{2}\NormalTok{)}\SpecialCharTok{\textasciitilde{}}\NormalTok{X1}\SpecialCharTok{+}\NormalTok{X2)}
\FunctionTok{bptest}\NormalTok{(Lmodel2)}
\end{Highlighting}
\end{Shaded}

\begin{verbatim}
## 
##  studentized Breusch-Pagan test
## 
## data:  Lmodel2
## BP = 5.0338, df = 2, p-value = 0.08071
\end{verbatim}

3.f)

Ho: Linear Model fits the Data(Y=b0+b1X1+b2X2) Ha: There is lack of fit
in the model(Y\textless\textgreater b0+b1X1+b2X2)

Since the P test value of X1 and X2 are greater than 0.01,so we reject
the null hypothesis.

\begin{Shaded}
\begin{Highlighting}[]
\CommentTok{\#Lmodel=lm(Y\textasciitilde{}X1+X2)}

\FunctionTok{anova}\NormalTok{(Lmodel2,Lmodel)}
\end{Highlighting}
\end{Shaded}

\begin{verbatim}
## Warning in anova.lmlist(object, ...): models with response '"Y"' removed because
## response differs from model 1
\end{verbatim}

\begin{verbatim}
## Analysis of Variance Table
## 
## Response: log((Lmodel$residuals)^2)
##           Df Sum Sq Mean Sq F value Pr(>F)
## X1         1 14.058 14.0580  3.1016 0.1017
## X2         1  0.202  0.2015  0.0445 0.8363
## Residuals 13 58.923  4.5325
\end{verbatim}

Problem 4

4.a)

Stem and leaf represents the histograms of quantitative data.

\begin{Shaded}
\begin{Highlighting}[]
\NormalTok{Data4}\OtherTok{=}\FunctionTok{read.table}\NormalTok{(}\StringTok{"As34Data.txt"}\NormalTok{, }\AttributeTok{header =} \ConstantTok{FALSE}\NormalTok{, }\AttributeTok{sep =} \StringTok{""}\NormalTok{) }
\NormalTok{Y }\OtherTok{=}\NormalTok{ Data4}\SpecialCharTok{$}\NormalTok{V1}
\NormalTok{X1 }\OtherTok{=}\NormalTok{ Data4}\SpecialCharTok{$}\NormalTok{V2}
\NormalTok{X2 }\OtherTok{=}\NormalTok{ Data4}\SpecialCharTok{$}\NormalTok{V3}
\NormalTok{X3 }\OtherTok{=}\NormalTok{ Data4}\SpecialCharTok{$}\NormalTok{V4}
\NormalTok{X4 }\OtherTok{=}\NormalTok{ Data4}\SpecialCharTok{$}\NormalTok{V5}
 
\NormalTok{Xmat}\OtherTok{=}\FunctionTok{as.matrix}\NormalTok{(}\FunctionTok{cbind}\NormalTok{(}\FunctionTok{rep}\NormalTok{(}\DecValTok{1}\NormalTok{,}\FunctionTok{length}\NormalTok{(Y)),X1,X2,X3,X4))}
\FunctionTok{stem}\NormalTok{(X1)}
\end{Highlighting}
\end{Shaded}

\begin{verbatim}
## 
##   The decimal point is at the |
## 
##    0 | 0000000000000000
##    2 | 00000000000000000000000
##    4 | 00000
##    6 | 0
##    8 | 0
##   10 | 00
##   12 | 00000
##   14 | 0000000000000
##   16 | 0000000000
##   18 | 000
##   20 | 00
\end{verbatim}

\begin{Shaded}
\begin{Highlighting}[]
\FunctionTok{stem}\NormalTok{(X2)}
\end{Highlighting}
\end{Shaded}

\begin{verbatim}
## 
##   The decimal point is at the |
## 
##    2 | 0
##    4 | 080003358
##    6 | 012613
##    8 | 00001223456001555689
##   10 | 013344566677778123344666668
##   12 | 00011115777889002
##   14 | 6
\end{verbatim}

\begin{Shaded}
\begin{Highlighting}[]
\FunctionTok{stem}\NormalTok{(X3)}
\end{Highlighting}
\end{Shaded}

\begin{verbatim}
## 
##   The decimal point is 1 digit(s) to the left of the |
## 
##   0 | 0000000000000000000000000000002333333333334444445555556678889
##   1 | 023444469
##   2 | 1223477
##   3 | 3
##   4 | 
##   5 | 7
##   6 | 0
##   7 | 3
\end{verbatim}

\begin{Shaded}
\begin{Highlighting}[]
\FunctionTok{stem}\NormalTok{(X4)}
\end{Highlighting}
\end{Shaded}

\begin{verbatim}
## 
##   The decimal point is 5 digit(s) to the right of the |
## 
##   0 | 333333444444
##   0 | 555666667778899
##   1 | 000001111222333334
##   1 | 578889
##   2 | 011122334444
##   2 | 555788899
##   3 | 002
##   3 | 567
##   4 | 23
##   4 | 8
\end{verbatim}

4.b)

From observing the correlation matrix we can see that there is no strong
correlation between the predictor and response variables.

\begin{Shaded}
\begin{Highlighting}[]
\FunctionTok{pairs}\NormalTok{(}\SpecialCharTok{\textasciitilde{}}\NormalTok{Y}\SpecialCharTok{+}\NormalTok{X1}\SpecialCharTok{+}\NormalTok{X2}\SpecialCharTok{+}\NormalTok{X3}\SpecialCharTok{+}\NormalTok{X4)}
\end{Highlighting}
\end{Shaded}

\includegraphics{Assn-03-Stats_files/figure-latex/unnamed-chunk-16-1.pdf}

\begin{Shaded}
\begin{Highlighting}[]
\FunctionTok{colnames}\NormalTok{(Data4)}\OtherTok{=}\FunctionTok{c}\NormalTok{(}\StringTok{"Y"}\NormalTok{,}\StringTok{"X1"}\NormalTok{,}\StringTok{"X2"}\NormalTok{,}\StringTok{"X3"}\NormalTok{,}\StringTok{"X4"}\NormalTok{)}
\FunctionTok{cor}\NormalTok{(Data4)}
\end{Highlighting}
\end{Shaded}

\begin{verbatim}
##              Y         X1         X2          X3         X4
## Y   1.00000000 -0.2502846  0.4137872  0.06652647 0.53526237
## X1 -0.25028456  1.0000000  0.3888264 -0.25266347 0.28858350
## X2  0.41378716  0.3888264  1.0000000 -0.37976174 0.44069713
## X3  0.06652647 -0.2526635 -0.3797617  1.00000000 0.08061073
## X4  0.53526237  0.2885835  0.4406971  0.08061073 1.00000000
\end{verbatim}

4.c)

The regression model is
Y=12.22-0.14\emph{X1+0.28}X2+0.61\emph{X3+7.92e-06}X4

\begin{Shaded}
\begin{Highlighting}[]
\NormalTok{Lmod}\OtherTok{=}\FunctionTok{lm}\NormalTok{(Y}\SpecialCharTok{\textasciitilde{}}\NormalTok{X1}\SpecialCharTok{+}\NormalTok{X2}\SpecialCharTok{+}\NormalTok{X3}\SpecialCharTok{+}\NormalTok{X4)}
\NormalTok{Lmod}
\end{Highlighting}
\end{Shaded}

\begin{verbatim}
## 
## Call:
## lm(formula = Y ~ X1 + X2 + X3 + X4)
## 
## Coefficients:
## (Intercept)           X1           X2           X3           X4  
##   1.220e+01   -1.420e-01    2.820e-01    6.193e-01    7.924e-06
\end{verbatim}

4.d)

There are some outliers in the data and the plot is noy symmetrical and
the assumption made on noise of regression is wrong.

\begin{Shaded}
\begin{Highlighting}[]
\NormalTok{Lmod}\SpecialCharTok{$}\NormalTok{residuals}
\end{Highlighting}
\end{Shaded}

\begin{verbatim}
##            1            2            3            4            5            6 
## -1.035672440 -1.513806414 -0.591053402 -0.133568082  0.313283765 -3.187185224 
##            7            8            9           10           11           12 
## -0.538356749  0.236302386  1.989220372  0.105829603  0.023124830 -0.337070751 
##           13           14           15           16           17           18 
##  0.717869468 -0.392411015 -0.201019573 -0.814937024  0.101690072 -1.759131637 
##           19           20           21           22           23           24 
## -1.210114916 -0.634341765 -0.366004170  0.288596123 -0.093200248  0.233884284 
##           25           26           27           28           29           30 
## -0.853339941 -2.123934469  0.466014057 -0.573974675 -1.068826727 -0.197717691 
##           31           32           33           34           35           36 
## -1.121737177 -0.173906919 -1.030125636 -0.090953654  0.215053952  0.784804746 
##           37           38           39           40           41           42 
##  1.083920373 -2.132451269 -0.185470952 -1.120385453 -0.012771680  2.500938643 
##           43           44           45           46           47           48 
## -1.582833452  0.929599530  0.394236721  0.117200255  0.815339787  1.605896564 
##           49           50           51           52           53           54 
##  0.557941960  0.494737472  0.207611404 -0.032045798  1.155796537  0.234272601 
##           55           56           57           58           59           60 
## -1.073489739  1.059646672 -0.261711555  1.031651273 -0.345957207  0.203372872 
##           61           62           63           64           65           66 
##  0.917961126  2.944144932  2.459696482  1.859088749  1.451807658 -0.483857748 
##           67           68           69           70           71           72 
## -0.756250356  2.011402309  0.078550427  0.009892809  1.766898426 -0.463930876 
##           73           74           75           76           77           78 
## -0.510410866 -0.106354746  1.209427169 -0.261085606 -0.627547725  0.910085787 
##           79           80           81 
## -0.550846871 -2.030180944 -0.906819056
\end{verbatim}

\begin{Shaded}
\begin{Highlighting}[]
\FunctionTok{boxplot}\NormalTok{(Lmod}\SpecialCharTok{$}\NormalTok{residuals)}
\end{Highlighting}
\end{Shaded}

\includegraphics{Assn-03-Stats_files/figure-latex/unnamed-chunk-18-1.pdf}
4.e)

Based on the residual plots we find that residuals are not uniform with
mean 0.

\begin{Shaded}
\begin{Highlighting}[]
\FunctionTok{par}\NormalTok{(}\AttributeTok{mfrow=}\FunctionTok{c}\NormalTok{(}\DecValTok{2}\NormalTok{,}\DecValTok{3}\NormalTok{))}
\FunctionTok{plot}\NormalTok{(Lmod}\SpecialCharTok{$}\NormalTok{fitted.values,Lmod}\SpecialCharTok{$}\NormalTok{residual)}
\FunctionTok{plot}\NormalTok{(X1,Lmod}\SpecialCharTok{$}\NormalTok{residual)}
\FunctionTok{plot}\NormalTok{(X2,Lmod}\SpecialCharTok{$}\NormalTok{residual)}
\FunctionTok{plot}\NormalTok{(X3,Lmod}\SpecialCharTok{$}\NormalTok{residual)}
\FunctionTok{plot}\NormalTok{(X4,Lmod}\SpecialCharTok{$}\NormalTok{residual)}

\FunctionTok{plot}\NormalTok{(X1,Lmod}\SpecialCharTok{$}\NormalTok{residual)}
\end{Highlighting}
\end{Shaded}

\includegraphics{Assn-03-Stats_files/figure-latex/unnamed-chunk-19-1.pdf}

\begin{Shaded}
\begin{Highlighting}[]
\FunctionTok{plot}\NormalTok{(X2,Lmod}\SpecialCharTok{$}\NormalTok{residual)}
\FunctionTok{plot}\NormalTok{(X3,Lmod}\SpecialCharTok{$}\NormalTok{residual)}
\FunctionTok{plot}\NormalTok{(X1}\SpecialCharTok{*}\NormalTok{X2,Lmod}\SpecialCharTok{$}\NormalTok{residual)}
\FunctionTok{plot}\NormalTok{(X2}\SpecialCharTok{*}\NormalTok{X3,Lmod}\SpecialCharTok{$}\NormalTok{residual)}
\FunctionTok{plot}\NormalTok{(X3}\SpecialCharTok{*}\NormalTok{X1,Lmod}\SpecialCharTok{$}\NormalTok{residual)}

\FunctionTok{plot}\NormalTok{(Lmod,}\AttributeTok{which=}\DecValTok{2}\NormalTok{)}
\end{Highlighting}
\end{Shaded}

\includegraphics{Assn-03-Stats_files/figure-latex/unnamed-chunk-19-2.pdf}

4.f)

From anova of Lmod we find that p value of coeficienr of X3 is less than
0.05 which implies X3 does not fit the model and X3=0.

The model can be Y\textasciitilde X1+X2+X4

\begin{Shaded}
\begin{Highlighting}[]
\FunctionTok{anova}\NormalTok{(Lmod)}\CommentTok{\#Y\textasciitilde{}X1+X2+X3+X4}
\end{Highlighting}
\end{Shaded}

\begin{verbatim}
## Analysis of Variance Table
## 
## Response: Y
##           Df Sum Sq Mean Sq F value    Pr(>F)    
## X1         1 14.819  14.819 11.4649  0.001125 ** 
## X2         1 72.802  72.802 56.3262 9.699e-11 ***
## X3         1  8.381   8.381  6.4846  0.012904 *  
## X4         1 42.325  42.325 32.7464 1.976e-07 ***
## Residuals 76 98.231   1.293                      
## ---
## Signif. codes:  0 '***' 0.001 '**' 0.01 '*' 0.05 '.' 0.1 ' ' 1
\end{verbatim}

\begin{Shaded}
\begin{Highlighting}[]
\NormalTok{Lmod2}\OtherTok{=}\FunctionTok{lm}\NormalTok{(Y}\SpecialCharTok{\textasciitilde{}}\NormalTok{X1}\SpecialCharTok{+}\NormalTok{X2}\SpecialCharTok{+}\NormalTok{X4)}

\FunctionTok{anova}\NormalTok{(Lmod2,Lmod)}
\end{Highlighting}
\end{Shaded}

\begin{verbatim}
## Analysis of Variance Table
## 
## Model 1: Y ~ X1 + X2 + X4
## Model 2: Y ~ X1 + X2 + X3 + X4
##   Res.Df    RSS Df Sum of Sq      F Pr(>F)
## 1     77 98.650                           
## 2     76 98.231  1   0.41975 0.3248 0.5704
\end{verbatim}

4.g)

H0: Error variance is constant Ha: Error variance is not constant

tstar\textgreater t so we reject null hypothesis. so the assumption
error variance is constant is true.

\begin{Shaded}
\begin{Highlighting}[]
\NormalTok{Yhat}\OtherTok{=}\FunctionTok{sort}\NormalTok{(}\FunctionTok{fitted}\NormalTok{(Lmod2))}

\NormalTok{Y1}\OtherTok{=}\NormalTok{Yhat[}\DecValTok{1}\SpecialCharTok{:}\DecValTok{40}\NormalTok{]}
\NormalTok{Y2}\OtherTok{=}\NormalTok{Yhat[}\DecValTok{41}\SpecialCharTok{:}\DecValTok{81}\NormalTok{]}

\NormalTok{e1}\OtherTok{=}\NormalTok{Y1}\SpecialCharTok{{-}}\NormalTok{Y[}\DecValTok{1}\SpecialCharTok{:}\DecValTok{40}\NormalTok{]}
\NormalTok{Med1}\OtherTok{=}\FunctionTok{median}\NormalTok{(e1)}

\NormalTok{e2}\OtherTok{=}\NormalTok{Y2}\SpecialCharTok{{-}}\NormalTok{Y[}\DecValTok{41}\SpecialCharTok{:}\DecValTok{81}\NormalTok{]}
\NormalTok{Med2}\OtherTok{=}\FunctionTok{median}\NormalTok{(e2)}

\NormalTok{n1}\OtherTok{=}\FunctionTok{length}\NormalTok{(Y1)}
\NormalTok{n2}\OtherTok{=}\FunctionTok{length}\NormalTok{(Y2)}


\NormalTok{d1}\OtherTok{=}\NormalTok{e1}\SpecialCharTok{{-}}\NormalTok{Med1}
\NormalTok{Mean1}\OtherTok{=}\FunctionTok{mean}\NormalTok{(d1)}
\NormalTok{d2}\OtherTok{=}\NormalTok{e2}\SpecialCharTok{{-}}\NormalTok{Med2}
\NormalTok{Mean2}\OtherTok{=}\FunctionTok{mean}\NormalTok{(d2)}

\NormalTok{s}\OtherTok{=}\FunctionTok{sqrt}\NormalTok{(}\FunctionTok{sum}\NormalTok{((d1}\SpecialCharTok{{-}}\NormalTok{Mean1)}\SpecialCharTok{\^{}}\DecValTok{2}\NormalTok{)}\SpecialCharTok{+}\FunctionTok{sum}\NormalTok{((d2}\SpecialCharTok{{-}}\NormalTok{Mean2)}\SpecialCharTok{\^{}}\DecValTok{2}\NormalTok{)}\SpecialCharTok{/}\NormalTok{(}\FunctionTok{length}\NormalTok{(Yhat)}\SpecialCharTok{{-}}\DecValTok{2}\NormalTok{))}

\NormalTok{tstar}\OtherTok{=}\NormalTok{(Mean1}\SpecialCharTok{{-}}\NormalTok{Mean2)}\SpecialCharTok{/}\NormalTok{(s}\SpecialCharTok{*}\FunctionTok{sqrt}\NormalTok{((}\DecValTok{1}\SpecialCharTok{/}\NormalTok{n1)}\SpecialCharTok{+}\NormalTok{(}\DecValTok{1}\SpecialCharTok{/}\NormalTok{n2)))}
\NormalTok{tstar}
\end{Highlighting}
\end{Shaded}

\begin{verbatim}
## [1] 0.09588115
\end{verbatim}

\begin{Shaded}
\begin{Highlighting}[]
\NormalTok{t}\OtherTok{=}\FunctionTok{qt}\NormalTok{(}\FloatTok{0.05}\NormalTok{,}\AttributeTok{df=}\NormalTok{n1}\SpecialCharTok{+}\NormalTok{n2}\DecValTok{{-}2}\NormalTok{)}
\NormalTok{t}
\end{Highlighting}
\end{Shaded}

\begin{verbatim}
## [1] -1.664371
\end{verbatim}

Problem 5

5.a)

H0: b1=b2=b3=b4=0 (all the coeficients are 0) Ha: At least one of the
coef is not 0

Since the p value of beta tests for X1, X2 , X3 and X4 are less than
0.05 we reject null hypothesis.

\begin{Shaded}
\begin{Highlighting}[]
\CommentTok{\#Q5}
\FunctionTok{anova}\NormalTok{(Lmod)}
\end{Highlighting}
\end{Shaded}

\begin{verbatim}
## Analysis of Variance Table
## 
## Response: Y
##           Df Sum Sq Mean Sq F value    Pr(>F)    
## X1         1 14.819  14.819 11.4649  0.001125 ** 
## X2         1 72.802  72.802 56.3262 9.699e-11 ***
## X3         1  8.381   8.381  6.4846  0.012904 *  
## X4         1 42.325  42.325 32.7464 1.976e-07 ***
## Residuals 76 98.231   1.293                      
## ---
## Signif. codes:  0 '***' 0.001 '**' 0.01 '*' 0.05 '.' 0.1 ' ' 1
\end{verbatim}

5.b)

The confidence intervals of betas are :

Beta1: (-0.19663959, -0.08742769) Beta2: (0.1203875, 0.4436456) Beta3:
(-2.161312, 3.399999) Beta4: (4.381297e-06, 1.146731e-05)

\begin{Shaded}
\begin{Highlighting}[]
\NormalTok{n}\OtherTok{=}\FunctionTok{length}\NormalTok{(Y)}
\NormalTok{alpha}\OtherTok{=}\DecValTok{1} \SpecialCharTok{{-}} \FloatTok{0.95}
\NormalTok{g}\OtherTok{=}\DecValTok{4}
\NormalTok{t}\OtherTok{=}\FunctionTok{qt}\NormalTok{(}\DecValTok{1} \SpecialCharTok{{-}}\NormalTok{ alpha}\SpecialCharTok{/}\NormalTok{(}\DecValTok{2} \SpecialCharTok{*}\NormalTok{ g), n }\SpecialCharTok{{-}} \DecValTok{4} \SpecialCharTok{{-}} \DecValTok{1}\NormalTok{)}
\NormalTok{t}
\end{Highlighting}
\end{Shaded}

\begin{verbatim}
## [1] 2.558541
\end{verbatim}

\begin{Shaded}
\begin{Highlighting}[]
\NormalTok{beta1 }\OtherTok{=} \FunctionTok{coef}\NormalTok{(}\FunctionTok{summary}\NormalTok{(Lmod))[,}\DecValTok{1}\NormalTok{][[}\DecValTok{2}\NormalTok{]]}
\NormalTok{beta2 }\OtherTok{=} \FunctionTok{coef}\NormalTok{(}\FunctionTok{summary}\NormalTok{(Lmod))[,}\DecValTok{1}\NormalTok{][[}\DecValTok{3}\NormalTok{]]}
\NormalTok{beta3 }\OtherTok{=} \FunctionTok{coef}\NormalTok{(}\FunctionTok{summary}\NormalTok{(Lmod))[,}\DecValTok{1}\NormalTok{][[}\DecValTok{4}\NormalTok{]]}
\NormalTok{beta4 }\OtherTok{=} \FunctionTok{coef}\NormalTok{(}\FunctionTok{summary}\NormalTok{(Lmod))[,}\DecValTok{1}\NormalTok{][[}\DecValTok{5}\NormalTok{]]}

\NormalTok{sebeta1 }\OtherTok{=} \FunctionTok{coef}\NormalTok{(}\FunctionTok{summary}\NormalTok{(Lmod))[,}\DecValTok{2}\NormalTok{][[}\DecValTok{2}\NormalTok{]]}
\NormalTok{sebeta2 }\OtherTok{=} \FunctionTok{coef}\NormalTok{(}\FunctionTok{summary}\NormalTok{(Lmod))[,}\DecValTok{2}\NormalTok{][[}\DecValTok{3}\NormalTok{]]}
\NormalTok{sebeta3 }\OtherTok{=} \FunctionTok{coef}\NormalTok{(}\FunctionTok{summary}\NormalTok{(Lmod))[,}\DecValTok{2}\NormalTok{][[}\DecValTok{4}\NormalTok{]]}
\NormalTok{sebeta4 }\OtherTok{=} \FunctionTok{coef}\NormalTok{(}\FunctionTok{summary}\NormalTok{(Lmod))[,}\DecValTok{2}\NormalTok{][[}\DecValTok{5}\NormalTok{]]}


\NormalTok{CIbeta1 }\OtherTok{=} \FunctionTok{c}\NormalTok{(beta1 }\SpecialCharTok{{-}}\NormalTok{ t }\SpecialCharTok{*}\NormalTok{ sebeta1, beta1 }\SpecialCharTok{+}\NormalTok{ t }\SpecialCharTok{*}\NormalTok{ sebeta1)}
\NormalTok{CIbeta2 }\OtherTok{=} \FunctionTok{c}\NormalTok{(beta2 }\SpecialCharTok{{-}}\NormalTok{ t }\SpecialCharTok{*}\NormalTok{ sebeta2, beta2 }\SpecialCharTok{+}\NormalTok{ t }\SpecialCharTok{*}\NormalTok{ sebeta2)}
\NormalTok{CIbeta3 }\OtherTok{=} \FunctionTok{c}\NormalTok{(beta3 }\SpecialCharTok{{-}}\NormalTok{ t }\SpecialCharTok{*}\NormalTok{ sebeta3, beta3 }\SpecialCharTok{+}\NormalTok{ t }\SpecialCharTok{*}\NormalTok{ sebeta3)}
\NormalTok{CIbeta4 }\OtherTok{=} \FunctionTok{c}\NormalTok{(beta4 }\SpecialCharTok{{-}}\NormalTok{ t }\SpecialCharTok{*}\NormalTok{ sebeta4, beta4 }\SpecialCharTok{+}\NormalTok{ t }\SpecialCharTok{*}\NormalTok{ sebeta4)}


\NormalTok{CIbeta1}
\end{Highlighting}
\end{Shaded}

\begin{verbatim}
## [1] -0.19663959 -0.08742769
\end{verbatim}

\begin{Shaded}
\begin{Highlighting}[]
\NormalTok{CIbeta2}
\end{Highlighting}
\end{Shaded}

\begin{verbatim}
## [1] 0.1203875 0.4436456
\end{verbatim}

\begin{Shaded}
\begin{Highlighting}[]
\NormalTok{CIbeta3}
\end{Highlighting}
\end{Shaded}

\begin{verbatim}
## [1] -2.161312  3.399999
\end{verbatim}

\begin{Shaded}
\begin{Highlighting}[]
\NormalTok{CIbeta4}
\end{Highlighting}
\end{Shaded}

\begin{verbatim}
## [1] 4.381297e-06 1.146731e-05
\end{verbatim}

5.c)

The value of Rsquare is 0.58 the variation of model explains 58\%
variation in Y wrt X.

\begin{Shaded}
\begin{Highlighting}[]
\NormalTok{sse }\OtherTok{=} \FunctionTok{sum}\NormalTok{((}\FunctionTok{fitted}\NormalTok{(Lmod) }\SpecialCharTok{{-}}\NormalTok{ Y)}\SpecialCharTok{\^{}}\DecValTok{2}\NormalTok{)}

\NormalTok{sst}\OtherTok{=}\FunctionTok{sum}\NormalTok{((Y}\SpecialCharTok{{-}}\FunctionTok{mean}\NormalTok{(Y))}\SpecialCharTok{\^{}}\DecValTok{2}\NormalTok{)}

\NormalTok{Rsquare}\OtherTok{=}\DecValTok{1}\SpecialCharTok{{-}}\NormalTok{(sse}\SpecialCharTok{/}\NormalTok{sst)}
\NormalTok{Rsquare}
\end{Highlighting}
\end{Shaded}

\begin{verbatim}
## [1] 0.5847496
\end{verbatim}

Problem 6

The family of estimates of coeficients is

\begin{Shaded}
\begin{Highlighting}[]
\CommentTok{\#Q6}

\NormalTok{Data42}\OtherTok{=}\FunctionTok{read.table}\NormalTok{(}\StringTok{"As36.txt"}\NormalTok{, }\AttributeTok{header =} \ConstantTok{FALSE}\NormalTok{, }\AttributeTok{sep =} \StringTok{""}\NormalTok{) }
\end{Highlighting}
\end{Shaded}

\begin{verbatim}
## Warning in read.table("As36.txt", header = FALSE, sep = ""): incomplete final
## line found by readTableHeader on 'As36.txt'
\end{verbatim}

\begin{Shaded}
\begin{Highlighting}[]
\NormalTok{n}\OtherTok{=}\FunctionTok{nrow}\NormalTok{(Data42)}
\NormalTok{Xh1}\OtherTok{=}\FunctionTok{t}\NormalTok{(}\FunctionTok{cbind}\NormalTok{(}\FunctionTok{rep}\NormalTok{(}\DecValTok{1}\NormalTok{,n),}\FunctionTok{t}\NormalTok{(Data42}\SpecialCharTok{$}\NormalTok{V1)))}
\end{Highlighting}
\end{Shaded}

\begin{verbatim}
## Warning in cbind(rep(1, n), t(Data42$V1)): number of rows of result is not a
## multiple of vector length (arg 1)
\end{verbatim}

\begin{Shaded}
\begin{Highlighting}[]
\NormalTok{Xh2}\OtherTok{=}\FunctionTok{t}\NormalTok{(}\FunctionTok{cbind}\NormalTok{(}\FunctionTok{rep}\NormalTok{(}\DecValTok{1}\NormalTok{,n),}\FunctionTok{t}\NormalTok{(Data42}\SpecialCharTok{$}\NormalTok{V2)))}
\end{Highlighting}
\end{Shaded}

\begin{verbatim}
## Warning in cbind(rep(1, n), t(Data42$V2)): number of rows of result is not a
## multiple of vector length (arg 1)
\end{verbatim}

\begin{Shaded}
\begin{Highlighting}[]
\NormalTok{Xh3}\OtherTok{=}\FunctionTok{t}\NormalTok{(}\FunctionTok{cbind}\NormalTok{(}\FunctionTok{rep}\NormalTok{(}\DecValTok{1}\NormalTok{,n),}\FunctionTok{t}\NormalTok{(Data42}\SpecialCharTok{$}\NormalTok{V3)))}
\end{Highlighting}
\end{Shaded}

\begin{verbatim}
## Warning in cbind(rep(1, n), t(Data42$V3)): number of rows of result is not a
## multiple of vector length (arg 1)
\end{verbatim}

\begin{Shaded}
\begin{Highlighting}[]
\NormalTok{Xh4}\OtherTok{=}\FunctionTok{t}\NormalTok{(}\FunctionTok{cbind}\NormalTok{(}\FunctionTok{rep}\NormalTok{(}\DecValTok{1}\NormalTok{,n),}\FunctionTok{t}\NormalTok{(Data42}\SpecialCharTok{$}\NormalTok{V4)))}
\end{Highlighting}
\end{Shaded}

\begin{verbatim}
## Warning in cbind(rep(1, n), t(Data42$V4)): number of rows of result is not a
## multiple of vector length (arg 1)
\end{verbatim}

\begin{Shaded}
\begin{Highlighting}[]
\NormalTok{beta0}\OtherTok{=}\FunctionTok{coef}\NormalTok{(}\FunctionTok{summary}\NormalTok{(Lmod))[,}\DecValTok{1}\NormalTok{][[}\DecValTok{2}\NormalTok{]]}

\NormalTok{Bmat}\OtherTok{=}\FunctionTok{cbind}\NormalTok{(beta0,beta1,beta2,beta3,beta4)}
\NormalTok{Bmat}\OtherTok{=}\FunctionTok{as.matrix}\NormalTok{(Bmat)}

\NormalTok{Yhat1}\OtherTok{=}\NormalTok{Bmat}\SpecialCharTok{\%*\%}\NormalTok{Xh1;}
\NormalTok{Yhat2}\OtherTok{=}\NormalTok{Bmat}\SpecialCharTok{\%*\%}\NormalTok{Xh2;}
\NormalTok{Yhat3}\OtherTok{=}\NormalTok{Bmat}\SpecialCharTok{\%*\%}\NormalTok{Xh3;}
\NormalTok{Yhat4}\OtherTok{=}\NormalTok{Bmat}\SpecialCharTok{\%*\%}\NormalTok{Xh4;}

\NormalTok{sse }\OtherTok{=} \FunctionTok{sum}\NormalTok{((}\FunctionTok{fitted}\NormalTok{(Lmod) }\SpecialCharTok{{-}}\NormalTok{ Y)}\SpecialCharTok{\^{}}\DecValTok{2}\NormalTok{)}

\NormalTok{Sigmasquare}\OtherTok{=}\NormalTok{sse}\SpecialCharTok{/}\NormalTok{n}\DecValTok{{-}2}

\NormalTok{measure1}\OtherTok{=}\NormalTok{(Sigmasquare}\SpecialCharTok{*}\FunctionTok{t}\NormalTok{(Xh1)}\SpecialCharTok{\%*\%}\FunctionTok{solve}\NormalTok{(}\FunctionTok{t}\NormalTok{(Xmat)}\SpecialCharTok{\%*\%}\NormalTok{Xmat)}\SpecialCharTok{\%*\%}\NormalTok{Xh1)}\SpecialCharTok{\^{}}\FloatTok{0.5}

\NormalTok{measure2}\OtherTok{=}\NormalTok{(Sigmasquare}\SpecialCharTok{*}\FunctionTok{t}\NormalTok{(Xh2)}\SpecialCharTok{\%*\%}\FunctionTok{solve}\NormalTok{(}\FunctionTok{t}\NormalTok{(Xmat)}\SpecialCharTok{\%*\%}\NormalTok{Xmat)}\SpecialCharTok{\%*\%}\NormalTok{Xh2)}\SpecialCharTok{\^{}}\FloatTok{0.5}

\NormalTok{measure3}\OtherTok{=}\NormalTok{(Sigmasquare}\SpecialCharTok{*}\FunctionTok{t}\NormalTok{(Xh3)}\SpecialCharTok{\%*\%}\FunctionTok{solve}\NormalTok{(}\FunctionTok{t}\NormalTok{(Xmat)}\SpecialCharTok{\%*\%}\NormalTok{Xmat)}\SpecialCharTok{\%*\%}\NormalTok{Xh3)}\SpecialCharTok{\^{}}\FloatTok{0.5}

\NormalTok{measure4}\OtherTok{=}\NormalTok{(Sigmasquare}\SpecialCharTok{*}\FunctionTok{t}\NormalTok{(Xh4)}\SpecialCharTok{\%*\%}\FunctionTok{solve}\NormalTok{(}\FunctionTok{t}\NormalTok{(Xmat)}\SpecialCharTok{\%*\%}\NormalTok{Xmat)}\SpecialCharTok{\%*\%}\NormalTok{Xh4)}\SpecialCharTok{\^{}}\FloatTok{0.5}

\NormalTok{W }\OtherTok{=} \FunctionTok{sqrt}\NormalTok{(}\DecValTok{2} \SpecialCharTok{*} \FunctionTok{qf}\NormalTok{(}\AttributeTok{p =} \FloatTok{0.95}\NormalTok{, }\AttributeTok{df1 =} \DecValTok{5}\NormalTok{, }\AttributeTok{df2 =} \FunctionTok{length}\NormalTok{(Y) }\SpecialCharTok{{-}} \DecValTok{5}\NormalTok{))}


\NormalTok{conf1up}\OtherTok{=}\NormalTok{Yhat1}\SpecialCharTok{+}\NormalTok{W}\SpecialCharTok{*}\NormalTok{measure1 }
\NormalTok{conf1lo}\OtherTok{=}\NormalTok{Yhat1}\SpecialCharTok{{-}}\NormalTok{W}\SpecialCharTok{*}\NormalTok{measure1  }
\NormalTok{conf2up}\OtherTok{=}\NormalTok{Yhat2}\SpecialCharTok{+}\NormalTok{W}\SpecialCharTok{*}\NormalTok{measure2 }
\NormalTok{conf2lo}\OtherTok{=}\NormalTok{Yhat2}\SpecialCharTok{{-}}\NormalTok{W}\SpecialCharTok{*}\NormalTok{measure2  }
\NormalTok{conf3up}\OtherTok{=}\NormalTok{Yhat3}\SpecialCharTok{+}\NormalTok{W}\SpecialCharTok{*}\NormalTok{measure3  }
\NormalTok{conf3lo}\OtherTok{=}\NormalTok{Yhat3}\SpecialCharTok{{-}}\NormalTok{W}\SpecialCharTok{*}\NormalTok{measure3  }
\NormalTok{conf4lo}\OtherTok{=}\NormalTok{Yhat3}\SpecialCharTok{+}\NormalTok{W}\SpecialCharTok{*}\NormalTok{measure3}
\NormalTok{conf4up}\OtherTok{=}\NormalTok{Yhat3}\SpecialCharTok{{-}}\NormalTok{W}\SpecialCharTok{*}\NormalTok{measure3}
\FunctionTok{c}\NormalTok{(conf1up,conf1lo)}
\end{Highlighting}
\end{Shaded}

\begin{verbatim}
## [1]  145.7784 -126.7568
\end{verbatim}

\begin{Shaded}
\begin{Highlighting}[]
\FunctionTok{c}\NormalTok{(conf2up,conf2lo)}
\end{Highlighting}
\end{Shaded}

\begin{verbatim}
## [1]  120.6433 -104.2315
\end{verbatim}

\begin{Shaded}
\begin{Highlighting}[]
\FunctionTok{c}\NormalTok{(conf3up,conf3lo)}
\end{Highlighting}
\end{Shaded}

\begin{verbatim}
## [1]  4.627043 -4.645127
\end{verbatim}

\begin{Shaded}
\begin{Highlighting}[]
\FunctionTok{c}\NormalTok{(conf4up,conf4lo)}
\end{Highlighting}
\end{Shaded}

\begin{verbatim}
## [1] -4.645127  4.627043
\end{verbatim}

Problem 7

7.a)

Transforming the data and fitting the model

\begin{Shaded}
\begin{Highlighting}[]
\NormalTok{Ycor  }\OtherTok{=} \FunctionTok{sqrt}\NormalTok{(}\DecValTok{1}\SpecialCharTok{/}\NormalTok{(}\FunctionTok{length}\NormalTok{(Y)}\SpecialCharTok{{-}}\DecValTok{1}\NormalTok{))}\SpecialCharTok{*}\NormalTok{((Y}\SpecialCharTok{{-}}\FunctionTok{mean}\NormalTok{(Y))}\SpecialCharTok{/}\FunctionTok{sd}\NormalTok{(Y))}
\NormalTok{X1cor }\OtherTok{=} \FunctionTok{sqrt}\NormalTok{(}\DecValTok{1}\SpecialCharTok{/}\NormalTok{(}\FunctionTok{length}\NormalTok{(X1)}\SpecialCharTok{{-}}\DecValTok{1}\NormalTok{))}\SpecialCharTok{*}\NormalTok{((X1}\SpecialCharTok{{-}}\FunctionTok{mean}\NormalTok{(X1))}\SpecialCharTok{/}\FunctionTok{sd}\NormalTok{(X1))}
\NormalTok{X2cor }\OtherTok{=} \FunctionTok{sqrt}\NormalTok{(}\DecValTok{1}\SpecialCharTok{/}\NormalTok{(}\FunctionTok{length}\NormalTok{(X2)}\SpecialCharTok{{-}}\DecValTok{1}\NormalTok{))}\SpecialCharTok{*}\NormalTok{((X2}\SpecialCharTok{{-}}\FunctionTok{mean}\NormalTok{(X2))}\SpecialCharTok{/}\FunctionTok{sd}\NormalTok{(X2))}
\NormalTok{X3cor }\OtherTok{=} \FunctionTok{sqrt}\NormalTok{(}\DecValTok{1}\SpecialCharTok{/}\NormalTok{(}\FunctionTok{length}\NormalTok{(X3)}\SpecialCharTok{{-}}\DecValTok{1}\NormalTok{))}\SpecialCharTok{*}\NormalTok{((X3}\SpecialCharTok{{-}}\FunctionTok{mean}\NormalTok{(X3))}\SpecialCharTok{/}\FunctionTok{sd}\NormalTok{(X3))}
\NormalTok{X4cor }\OtherTok{=} \FunctionTok{sqrt}\NormalTok{(}\DecValTok{1}\SpecialCharTok{/}\NormalTok{(}\FunctionTok{length}\NormalTok{(X4)}\SpecialCharTok{{-}}\DecValTok{1}\NormalTok{))}\SpecialCharTok{*}\NormalTok{((X4}\SpecialCharTok{{-}}\FunctionTok{mean}\NormalTok{(X4))}\SpecialCharTok{/}\FunctionTok{sd}\NormalTok{(X4))}

\NormalTok{Lmodel}\OtherTok{=}\FunctionTok{lm}\NormalTok{(Ycor}\SpecialCharTok{\textasciitilde{}{-}}\DecValTok{1}\SpecialCharTok{+}\NormalTok{X1cor}\SpecialCharTok{+}\NormalTok{X2cor}\SpecialCharTok{+}\NormalTok{X3cor}\SpecialCharTok{+}\NormalTok{X4cor)}
\NormalTok{Lmodel}
\end{Highlighting}
\end{Shaded}

\begin{verbatim}
## 
## Call:
## lm(formula = Ycor ~ -1 + X1cor + X2cor + X3cor + X4cor)
## 
## Coefficients:
##    X1cor     X2cor     X3cor     X4cor  
## -0.54785   0.42365   0.04846   0.50276
\end{verbatim}

\begin{Shaded}
\begin{Highlighting}[]
\NormalTok{Lmod}\CommentTok{\#Y\textasciitilde{}X1+X2+X3+X4}
\end{Highlighting}
\end{Shaded}

\begin{verbatim}
## 
## Call:
## lm(formula = Y ~ X1 + X2 + X3 + X4)
## 
## Coefficients:
## (Intercept)           X1           X2           X3           X4  
##   1.220e+01   -1.420e-01    2.820e-01    6.193e-01    7.924e-06
\end{verbatim}

7.b)

The Standardization coef beta hat after transformation becomes :

Betahat2=Sy/Sk*beta2hatstar

\begin{Shaded}
\begin{Highlighting}[]
\NormalTok{Betahat2}\OtherTok{=}\NormalTok{(}\FunctionTok{sd}\NormalTok{(Y)}\SpecialCharTok{/}\FunctionTok{sd}\NormalTok{(X2))}\SpecialCharTok{*}\FloatTok{0.423}\CommentTok{\#value of beta from Lmodel}
\NormalTok{Betahat2}
\end{Highlighting}
\end{Shaded}

\begin{verbatim}
## [1] 0.2815859
\end{verbatim}

7.c)

The Standardization coef beta hat after transformatio becomes :

Betahatk=Sy/Sk*betakhatstar

\begin{Shaded}
\begin{Highlighting}[]
\NormalTok{Betahat1}\OtherTok{=}\NormalTok{(}\FunctionTok{sd}\NormalTok{(Y)}\SpecialCharTok{/}\FunctionTok{sd}\NormalTok{(X1))}\SpecialCharTok{*{-}}\FloatTok{0.547}
\NormalTok{Betahat2}\OtherTok{=}\NormalTok{(}\FunctionTok{sd}\NormalTok{(Y)}\SpecialCharTok{/}\FunctionTok{sd}\NormalTok{(X2))}\SpecialCharTok{*}\FloatTok{0.423}
\NormalTok{Betahat3}\OtherTok{=}\NormalTok{(}\FunctionTok{sd}\NormalTok{(Y)}\SpecialCharTok{/}\FunctionTok{sd}\NormalTok{(X3))}\SpecialCharTok{*}\FloatTok{0.048}
\NormalTok{Betahat4}\OtherTok{=}\NormalTok{(}\FunctionTok{sd}\NormalTok{(Y)}\SpecialCharTok{/}\FunctionTok{sd}\NormalTok{(X4))}\SpecialCharTok{*}\FloatTok{0.502}

\NormalTok{Betahat1}
\end{Highlighting}
\end{Shaded}

\begin{verbatim}
## [1] -0.1418126
\end{verbatim}

\begin{Shaded}
\begin{Highlighting}[]
\NormalTok{Betahat2}
\end{Highlighting}
\end{Shaded}

\begin{verbatim}
## [1] 0.2815859
\end{verbatim}

\begin{Shaded}
\begin{Highlighting}[]
\NormalTok{Betahat3}
\end{Highlighting}
\end{Shaded}

\begin{verbatim}
## [1] 0.6134472
\end{verbatim}

\begin{Shaded}
\begin{Highlighting}[]
\NormalTok{Betahat4}
\end{Highlighting}
\end{Shaded}

\begin{verbatim}
## [1] 7.912368e-06
\end{verbatim}

Problem 8

8.a)

The regression model is Y=50.775 + 4.425X1

\begin{Shaded}
\begin{Highlighting}[]
\NormalTok{Y }\OtherTok{=}\NormalTok{ Data3}\SpecialCharTok{$}\NormalTok{Y}
\NormalTok{X1}\OtherTok{=}\NormalTok{ Data3}\SpecialCharTok{$}\NormalTok{X1}
\NormalTok{X2}\OtherTok{=}\NormalTok{Data3}\SpecialCharTok{$}\NormalTok{X2}

\NormalTok{Linearmod}\OtherTok{=}\FunctionTok{lm}\NormalTok{(Y}\SpecialCharTok{\textasciitilde{}}\NormalTok{X1)}
\NormalTok{Linearmod}
\end{Highlighting}
\end{Shaded}

\begin{verbatim}
## 
## Call:
## lm(formula = Y ~ X1)
## 
## Coefficients:
## (Intercept)           X1  
##      50.775        4.425
\end{verbatim}

8.b)

We have observed that the coefficient of moisture in model 6.5b is equal
to that of the moisture coefficient in this model

\begin{Shaded}
\begin{Highlighting}[]
\FunctionTok{lm}\NormalTok{(Y}\SpecialCharTok{\textasciitilde{}}\NormalTok{X1}\SpecialCharTok{+}\NormalTok{X2)}\CommentTok{\#model in 6.5b}
\end{Highlighting}
\end{Shaded}

\begin{verbatim}
## 
## Call:
## lm(formula = Y ~ X1 + X2)
## 
## Coefficients:
## (Intercept)           X1           X2  
##      37.650        4.425        4.375
\end{verbatim}

8.c)

From anova of the models SSR(X1\textbar X2) = SSR(X1)

\begin{Shaded}
\begin{Highlighting}[]
\FunctionTok{anova}\NormalTok{(Linearmod)}
\end{Highlighting}
\end{Shaded}

\begin{verbatim}
## Analysis of Variance Table
## 
## Response: Y
##           Df  Sum Sq Mean Sq F value    Pr(>F)    
## X1         1 1566.45 1566.45  54.751 3.356e-06 ***
## Residuals 14  400.55   28.61                      
## ---
## Signif. codes:  0 '***' 0.001 '**' 0.01 '*' 0.05 '.' 0.1 ' ' 1
\end{verbatim}

\begin{Shaded}
\begin{Highlighting}[]
\FunctionTok{anova}\NormalTok{(}\FunctionTok{lm}\NormalTok{(Y}\SpecialCharTok{\textasciitilde{}}\NormalTok{X1}\SpecialCharTok{+}\NormalTok{X2))}
\end{Highlighting}
\end{Shaded}

\begin{verbatim}
## Analysis of Variance Table
## 
## Response: Y
##           Df  Sum Sq Mean Sq F value    Pr(>F)    
## X1         1 1566.45 1566.45 215.947 1.778e-09 ***
## X2         1  306.25  306.25  42.219 2.011e-05 ***
## Residuals 13   94.30    7.25                      
## ---
## Signif. codes:  0 '***' 0.001 '**' 0.01 '*' 0.05 '.' 0.1 ' ' 1
\end{verbatim}

\begin{Shaded}
\begin{Highlighting}[]
\CommentTok{\#SSR X1/X2= = SSR(X1, X2) − SSR(X2)}
\NormalTok{Ssrx1\_x2}\OtherTok{=}\FloatTok{1566.45+306.25{-}306.25}
\NormalTok{Ssrx1\_x2}
\end{Highlighting}
\end{Shaded}

\begin{verbatim}
## [1] 1566.45
\end{verbatim}

8.d)

Based on (b) and (c), and also the correlation matrix in Problem 6.5(a)
confirms that there is a strong linear relationship between response
variable and moisture content X1.

Problem 9

9.a)

By plotting the graph the relation did not appear the same for both the
populations.

\begin{Shaded}
\begin{Highlighting}[]
\NormalTok{Data9}\OtherTok{=}\FunctionTok{read.table}\NormalTok{(}\StringTok{"AS39Data.txt"}\NormalTok{, }\AttributeTok{header =} \ConstantTok{FALSE}\NormalTok{, }\AttributeTok{sep =} \StringTok{""}\NormalTok{) }
\NormalTok{Y }\OtherTok{=}\NormalTok{ Data9}\SpecialCharTok{$}\NormalTok{V1}
\NormalTok{X1 }\OtherTok{=}\NormalTok{ Data9}\SpecialCharTok{$}\NormalTok{V2}
\NormalTok{X2}\OtherTok{=}\NormalTok{Data9}\SpecialCharTok{$}\NormalTok{V3}

\FunctionTok{library}\NormalTok{(ggplot2)}
\FunctionTok{ggplot}\NormalTok{(Data9, }\FunctionTok{aes}\NormalTok{(X1, Y, }\AttributeTok{colour =} \FunctionTok{as.factor}\NormalTok{(X2))) }\SpecialCharTok{+} \FunctionTok{geom\_point}\NormalTok{()}
\end{Highlighting}
\end{Shaded}

\includegraphics{Assn-03-Stats_files/figure-latex/unnamed-chunk-32-1.pdf}

\begin{Shaded}
\begin{Highlighting}[]
\FunctionTok{plot}\NormalTok{(Y,X1}\SpecialCharTok{+}\NormalTok{X2)}
\end{Highlighting}
\end{Shaded}

\includegraphics{Assn-03-Stats_files/figure-latex/unnamed-chunk-32-2.pdf}

9.b)

H0:All the coeficients are zero Ha : At least one of the coefficient is
not zero

since, Fstar \textgreater{} F-ratio i.e 18.65\textgreater3.15,
therefore, we reject null hypothesis.

\begin{Shaded}
\begin{Highlighting}[]
\NormalTok{fit }\OtherTok{=} \FunctionTok{lm}\NormalTok{(Y}\SpecialCharTok{\textasciitilde{}}\NormalTok{X1}\SpecialCharTok{+}\NormalTok{X2}\SpecialCharTok{+}\NormalTok{X1}\SpecialCharTok{*}\NormalTok{X2)}
\NormalTok{fit}
\end{Highlighting}
\end{Shaded}

\begin{verbatim}
## 
## Call:
## lm(formula = Y ~ X1 + X2 + X1 * X2)
## 
## Coefficients:
## (Intercept)           X1           X2        X1:X2  
##    -126.905        2.776       76.022       -1.107
\end{verbatim}

\begin{Shaded}
\begin{Highlighting}[]
\FunctionTok{summary}\NormalTok{(fit)}
\end{Highlighting}
\end{Shaded}

\begin{verbatim}
## 
## Call:
## lm(formula = Y ~ X1 + X2 + X1 * X2)
## 
## Residuals:
##      Min       1Q   Median       3Q      Max 
## -10.8470  -2.1639   0.0913   1.9348   9.9836 
## 
## Coefficients:
##              Estimate Std. Error t value Pr(>|t|)    
## (Intercept) -126.9052    14.7225  -8.620 4.33e-12 ***
## X1             2.7759     0.1963  14.142  < 2e-16 ***
## X2            76.0215    30.1314   2.523  0.01430 *  
## X1:X2         -1.1075     0.4055  -2.731  0.00828 ** 
## ---
## Signif. codes:  0 '***' 0.001 '**' 0.01 '*' 0.05 '.' 0.1 ' ' 1
## 
## Residual standard error: 3.893 on 60 degrees of freedom
## Multiple R-squared:  0.8233, Adjusted R-squared:  0.8145 
## F-statistic: 93.21 on 3 and 60 DF,  p-value: < 2.2e-16
\end{verbatim}

\begin{Shaded}
\begin{Highlighting}[]
\NormalTok{fit}\SpecialCharTok{$}\NormalTok{coef}
\end{Highlighting}
\end{Shaded}

\begin{verbatim}
## (Intercept)          X1          X2       X1:X2 
## -126.905171    2.775898   76.021532   -1.107482
\end{verbatim}

\begin{Shaded}
\begin{Highlighting}[]
\FunctionTok{anova}\NormalTok{(fit)}
\end{Highlighting}
\end{Shaded}

\begin{verbatim}
## Analysis of Variance Table
## 
## Response: Y
##           Df Sum Sq Mean Sq  F value    Pr(>F)    
## X1         1 3670.9  3670.9 242.2760 < 2.2e-16 ***
## X2         1  453.1   453.1  29.9073 9.282e-07 ***
## X1:X2      1  113.0   113.0   7.4578  0.008281 ** 
## Residuals 60  909.1    15.2                       
## ---
## Signif. codes:  0 '***' 0.001 '**' 0.01 '*' 0.05 '.' 0.1 ' ' 1
\end{verbatim}

\begin{Shaded}
\begin{Highlighting}[]
\NormalTok{fit }\OtherTok{=} \FunctionTok{lm}\NormalTok{(Y}\SpecialCharTok{\textasciitilde{}}\NormalTok{X1)}
\FunctionTok{anova}\NormalTok{(fit)}
\end{Highlighting}
\end{Shaded}

\begin{verbatim}
## Analysis of Variance Table
## 
## Response: Y
##           Df Sum Sq Mean Sq F value    Pr(>F)    
## X1         1 3670.9  3670.9  154.28 < 2.2e-16 ***
## Residuals 62 1475.3    23.8                      
## ---
## Signif. codes:  0 '***' 0.001 '**' 0.01 '*' 0.05 '.' 0.1 ' ' 1
\end{verbatim}

\begin{Shaded}
\begin{Highlighting}[]
\NormalTok{fit1 }\OtherTok{=} \FunctionTok{update}\NormalTok{(fit,.}\SpecialCharTok{\textasciitilde{}}\NormalTok{.}\SpecialCharTok{+}\NormalTok{X2}\SpecialCharTok{+}\NormalTok{X1}\SpecialCharTok{*}\NormalTok{X2)}
\FunctionTok{anova}\NormalTok{(fit1)}
\end{Highlighting}
\end{Shaded}

\begin{verbatim}
## Analysis of Variance Table
## 
## Response: Y
##           Df Sum Sq Mean Sq  F value    Pr(>F)    
## X1         1 3670.9  3670.9 242.2760 < 2.2e-16 ***
## X2         1  453.1   453.1  29.9073 9.282e-07 ***
## X1:X2      1  113.0   113.0   7.4578  0.008281 ** 
## Residuals 60  909.1    15.2                       
## ---
## Signif. codes:  0 '***' 0.001 '**' 0.01 '*' 0.05 '.' 0.1 ' ' 1
\end{verbatim}

\begin{Shaded}
\begin{Highlighting}[]
\FunctionTok{nrow}\NormalTok{(Data)}
\end{Highlighting}
\end{Shaded}

\begin{verbatim}
## [1] 10
\end{verbatim}

\begin{Shaded}
\begin{Highlighting}[]
\CommentTok{\#From anova table}

\CommentTok{\#SSR(X\_2, X\_1X\_2|X\_1) = SSR(X\_2, X\_1X\_2, X\_1) {-} SSR(X\_1)}
\CommentTok{\#=3670.9 + 453.1 + 113.0 {-} 3670.9}

\NormalTok{SSR}\OtherTok{=}\FloatTok{3670.9} \SpecialCharTok{+} \FloatTok{453.1} \SpecialCharTok{+} \FloatTok{113.0} \SpecialCharTok{{-}} \FloatTok{3670.9}

\NormalTok{MSEf}\OtherTok{=}\FloatTok{909.1}\SpecialCharTok{/}\DecValTok{60}

\NormalTok{DofF}\OtherTok{=}\DecValTok{3}
\NormalTok{DofP}\OtherTok{=}\DecValTok{1}

\NormalTok{Fstar}\OtherTok{=}\NormalTok{(SSR}\SpecialCharTok{/}\NormalTok{(DofF}\SpecialCharTok{{-}}\NormalTok{DofP))}\SpecialCharTok{/}\NormalTok{(}\FloatTok{909.1}\SpecialCharTok{/}\DecValTok{60}\NormalTok{)}
\NormalTok{Fstar}
\end{Highlighting}
\end{Shaded}

\begin{verbatim}
## [1] 18.68111
\end{verbatim}

\begin{Shaded}
\begin{Highlighting}[]
\FunctionTok{qf}\NormalTok{(}\FloatTok{0.95}\NormalTok{,}\DecValTok{2}\NormalTok{,}\DecValTok{60}\NormalTok{)}
\end{Highlighting}
\end{Shaded}

\begin{verbatim}
## [1] 3.150411
\end{verbatim}

9.c)

The nature of difference between two models is linear that is
Y=76.021+1.102X

\begin{Shaded}
\begin{Highlighting}[]
\NormalTok{Y11}\OtherTok{=}\NormalTok{Y[X2}\SpecialCharTok{==}\DecValTok{1}\NormalTok{]}
\NormalTok{Y12}\OtherTok{=}\NormalTok{Y[X2}\SpecialCharTok{==}\DecValTok{0}\NormalTok{]}
\NormalTok{X11}\OtherTok{=}\NormalTok{X1[X2}\SpecialCharTok{==}\DecValTok{1}\NormalTok{]}
\NormalTok{X10}\OtherTok{=}\NormalTok{X1[X2}\SpecialCharTok{==}\DecValTok{0}\NormalTok{]}

\NormalTok{LinMod1}\OtherTok{=}\FunctionTok{lm}\NormalTok{(Y11}\SpecialCharTok{\textasciitilde{}}\NormalTok{X11)}
\NormalTok{LinMod2}\OtherTok{=}\FunctionTok{lm}\NormalTok{(Y12}\SpecialCharTok{\textasciitilde{}}\NormalTok{X10)}

\NormalTok{LinMod1}
\end{Highlighting}
\end{Shaded}

\begin{verbatim}
## 
## Call:
## lm(formula = Y11 ~ X11)
## 
## Coefficients:
## (Intercept)          X11  
##     -50.884        1.668
\end{verbatim}

\begin{Shaded}
\begin{Highlighting}[]
\CommentTok{\#Y={-}50.884+1.668X}
\NormalTok{LinMod2}
\end{Highlighting}
\end{Shaded}

\begin{verbatim}
## 
## Call:
## lm(formula = Y12 ~ X10)
## 
## Coefficients:
## (Intercept)          X10  
##    -126.905        2.776
\end{verbatim}

\begin{Shaded}
\begin{Highlighting}[]
\CommentTok{\#Y={-}126.905+2.776X}

\CommentTok{\#Difference in the Model}
\CommentTok{\#Y=76.021+1.102X}

\FunctionTok{ggplot}\NormalTok{(Data9, }\FunctionTok{aes}\NormalTok{(}\AttributeTok{x=}\NormalTok{X1, }\AttributeTok{y=}\NormalTok{Y, }\AttributeTok{col=}\FunctionTok{as.factor}\NormalTok{(X2))) }\SpecialCharTok{+} \FunctionTok{geom\_point}\NormalTok{() }\SpecialCharTok{+}
            \FunctionTok{geom\_smooth}\NormalTok{(}\AttributeTok{method=}\StringTok{"lm"}\NormalTok{, }\AttributeTok{se=}\ConstantTok{FALSE}\NormalTok{)}
\end{Highlighting}
\end{Shaded}

\begin{verbatim}
## `geom_smooth()` using formula 'y ~ x'
\end{verbatim}

\includegraphics{Assn-03-Stats_files/figure-latex/unnamed-chunk-34-1.pdf}

\end{document}
